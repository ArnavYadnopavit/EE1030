%iffalse
\let\negmedspace\undefined
\let\negthickspace\undefined
\documentclass[journal,12pt,onecolumn]{IEEEtran}
\usepackage{cite}
\usepackage{amsmath,amssymb,amsfonts,amsthm}
\usepackage{algorithmic}
\usepackage{graphicx}
\usepackage{textcomp}
\usepackage{xcolor}
\usepackage{txfonts}
\usepackage{listings}
\usepackage{enumitem}
\usepackage{mathtools}
\usepackage{gensymb}
\usepackage{comment}
\usepackage[breaklinks=true]{hyperref}
\usepackage{tkz-euclide} 
\usepackage{listings}
\usepackage{gvv}                                        
%\def\inputGnumericTable{}                                 
\usepackage[latin1]{inputenc}     
\usepackage{xparse}
\usepackage{color}                                            
\usepackage{array}                                            
\usepackage{longtable}                                       
\usepackage{calc}                                             
\usepackage{multirow}
\usepackage{multicol}
\usepackage{hhline}                                           
\usepackage{ifthen}                                           
\usepackage{lscape}
\usepackage{tabularx}
\usepackage{array}
\usepackage{float}
\newtheorem{theorem}{Theorem}[section]
\newtheorem{problem}{Problem}
\newtheorem{proposition}{Proposition}[section]
\newtheorem{lemma}{Lemma}[section]
\newtheorem{corollary}[theorem]{Corollary}
\newtheorem{example}{Example}[section]
\newtheorem{definition}[problem]{Definition}
\newcommand{\BEQA}{\begin{eqnarray}}
\newcommand{\EEQA}{\end{eqnarray}}
\usepackage{float}
\usepackage{listings}
\usepackage{xcolor}
%\newcommand{\define}{\stackrel{\triangle}{=}}
\theoremstyle{remark}
\usepackage{ circuitikz }
%\newtheorem{rem}{Remark}
% Marks the beginning of the document
\begin{document}
\title{Software Assignment: Finding Eigenvalues}
\author{EE24BTECH11007 - Arnav Makarand Yadnopavit}
\maketitle
\renewcommand{\thefigure}{\theenumi}
\renewcommand{\thetable}{\theenumi}

\section{Introduction}
\parindent 0px There are various algorithms present to approximate eigenvalues of matrices. The best algorithms considering general $n\times n$ matrix are as following.\\

\begin{table}[h]
    \centering
    \begin{tabular}{|c|c|c|c|c|c|c|c|c|}
\hline
$t$ (hour) & 0 & 1 & 2 & 3 & 4 & 5 & 6 & 7 \\ \hline
$Q \brak{m^3/s}$ & 0 & 9 & 21 & 18 & 12 & 5 & 2 & 0 \\ \hline
\end{tabular}
\end{table}
Hence QR Algorithm with Householder technique and Wilkinson Shift stands out for being generally better for any general $n\times n$ matrix due to its fast convergence rate, Numerical stability.
\section{QR Algorithm with Householder technique and Wilkinson Shift}
\subsection{QR Decomposition}
QR decomposition factors a given matrix $A$ into:
$$A = QR,$$
where $Q$ is an orthogonal matrix $\brak{Q^\top Q = I}$, and $R$ is an upper triangular matrix.

\subsection{Householder Transformations}
Householder transformations are used to zero out elements below the diagonal of a matrix column. Given a vector $v$, the Householder matrix is:
$$H = I - 2\frac{vv^\top}{v^\top v}$$
The way it works is:\\
Initialize Q as Identity matrix. Let $\vec{x}$ be the the first column of $A$, and $\alpha=\norm{\vec{x}}$.\\
\begin{align*}
    \vec{u}&=\vec{x}-\alpha\vec{e_1}\\
    \vec{v}&=\frac{\vec{u}}{\norm{\vec{u}}}\\
    Q&=I-2vv^H
\end{align*}
By this we obtain $Q_1$ such that:\\
$$Q_1A=\myvec{\alpha_1 & * & \dots & *\\ 0 & & &\\\vdots & & A^\prime &\\0 & & &}$$
This can be repeated for $A^\prime$ (obtained from $Q_1A$ by deleting the first row and first column), resulting in a Householder matrix $Q^\prime_2$. Note that $Q\prime_2$ is smaller than $Q_1$. Since we want it really to operate on $Q_1A$ instead of $A^\prime$ we need to expand it to the upper left, filling in a 1, or in general:
$$Q_k=\myvec{I_{k-1} & 0 \\ 0 & Q^\prime_k}$$\\
After $n-1$ iterations of this process.\\
$R=Q_{n-1}\dots Q_2Q_1A$\\
$Q^\top=Q_{n-1}\dots Q_2Q_1$\\
$Q=Q_1Q_2\dots Q_{n-1}$
\subsection{QR Algorithm for Eigenvalues}
The QR algorithm iteratively applies QR decomposition to a shifted matrix \( A - \mu I \) and reconstructs it as:
$$A = RQ + \mu I$$
converging to an upper triangular form with eigenvalues on the diagonal.\\
where $\mu$ can be calculated by:\\
$$\mu=a_m-\frac{\delta}{\abs{\delta}}\frac{b^2_{m-1}}{\abs{\delta}+\sqrt{\delta^2+b^2_{m-1}}}$$
where B is the lower rightmost $2\times 2$ matrix of $A$, B= \myvec{a_{m-1}&b^\prime_{m-1}\\b^\prime_{m-1}&a_m}\\
$\delta=\frac{a_{m-1}-a_m}{2}$\\
If $\delta=0$, then $\mu=a_m-b_{m-1}$
\subsection{Complex Eigenvalues}
In case a matrix has complex eigenvalues a hessenberg matrix $\brak{2\times2}$ will be formed along the diagonal of the triangularised matrix A such that:
$$A=\myvec{\lambda_1 & \dots & \dots & \dots \\ 0 & a & b & \dots\\ \vdots & c & d & \dots\\ 0 & \dots & \dots & \lambda_n}$$
then:
\begin{align*}
    \lambda_2=\frac{a+d+\sqrt{\brak{a+d}^2-4\brak{ad-bc}}}{2}\\
    \lambda_3=\frac{a+d-\sqrt{\brak{a+d}^2-4\brak{ad-bc}}}{2}\\
\end{align*}

\section{Code}
\begin{verbatim}
#include <stdio.h>
#include <math.h>
#include <complex.h>
#include <stdlib.h>
//TO DO LIST
//QR DECOMPOSITION DONE
//QR Algo DONE
//MATRIX SUB DONE
//MATRIX SCALARMULT DONE
//MATRIX MULT DONE
//WILKINSON SHIFT DONE
//VECTOR NORM DONE
#define MAX_ITER 10000
#define ORDER 3
typedef struct matrix{
	double complex mat[ORDER][ORDER];
}matrix;
typedef struct QR{
	matrix Q;
	matrix R;
}QR;
double complex VectorInnerProduct(double complex* vector1,double complex* vector2){
	double complex ip=0;
	for (int i=0;i<ORDER;i++){
                ip+=vector1[i]*conj(vector2[i]);
        }
	return ip;
}
double complex VectorNorm(double complex* vector){
	return csqrt(VectorInnerProduct(vector,vector));
}
matrix Identity(){
	matrix Id;
	
	for (int i=0;i<ORDER;i++){
		for(int j=0;j<ORDER;j++){
			if (i==j){
				Id.mat[i][j]=1;
			}
			else{
				Id.mat[i][j]=0;
			}
		}
	}
	return Id;
}

matrix MatScalMult(matrix mat,double complex scal){
	matrix result;
	for (int i=0;i<ORDER;i++){
		for (int j=0;j<ORDER;j++){
			result.mat[i][j]=mat.mat[i][j]*scal;
		}
	}
	return result;
}
matrix MatSub(matrix mat1,matrix mat2){
	matrix result;

    for (int i=0;i<ORDER;i++){
            for (int j=0;j<ORDER;j++){
                result.mat[i][j]=mat1.mat[i][j]-mat2.mat[i][j];
            }       
    }       
        return result;
}
matrix MatMult(matrix mat1,matrix mat2) {
    matrix result;
    for (int i=0;i<ORDER;i++){
        for (int j=0;j<ORDER;j++){
            result.mat[i][j]=0;
        }
    }
    for (int i=0;i<ORDER;i++){
        for (int j=0;j<ORDER;j++){
            for (int k=0;k<ORDER;k++){
                result.mat[i][j]+=mat1.mat[i][k]*mat2.mat[k][j];
            }
        }
    }
    
    return result;
}
matrix trans(matrix mat){
    matrix result;
    for (int i=0;i<ORDER;i++){
        for (int j=0;j<ORDER;j++){
        result.mat[i][j]=conj(mat.mat[j][i]);
        }
    }
    return result;
}
double complex WilkinsonShift(matrix A){
    double complex delta = (A.mat[ORDER-2][ORDER-2]-A.mat[ORDER-1][ORDER-1])/2;
    if (cabs(delta)==0 && cabs(A.mat[ORDER-1][ORDER-2])==0){
        return A.mat[ORDER-1][ORDER-1]+1;
    }
    else if (cabs(delta)==0){
        return A.mat[ORDER-1][ORDER-1]-A.mat[ORDER-1][ORDER-2]+1;
    }
    else if(cabs(A.mat[ORDER-1][ORDER-2])==0){
        return A.mat[ORDER-1][ORDER-1]+1;
    }
    else{
        return A.mat[ORDER-1][ORDER-1]-((delta/cabs(delta))*(A.mat[ORDER-1][ORDER-2]*A.mat[ORDER-1][ORDER-2])/(cabs(delta)+csqrt(delta*delta+A.mat[ORDER-1][ORDER-2]*A.mat[ORDER-1][ORDER-2])));
    }
}
int tolcheck(matrix A){
    int t=0;
    double complex tol=1e-12;
    for (int i=1;i<ORDER;i++){
        for (int j=0;j<i;j++){
            if (cabs(A.mat[i][j])>cabs(tol)){
                t++;
                break;
            }
        }
    }
    if (t>0){
        return 1;
    } 
    else{return 0;}
}

QR QRDecomposition(matrix A){
    QR result;
    matrix  H;
    result.Q=Identity();
    result.R=A;    

    for (int k=0;k<ORDER-1;k++){
        double complex v[ORDER-k];
        double complex beta;
        for (int i=k;i<ORDER;i++){
            v[i-k]=result.R.mat[i][k];
        }
        double complex alpha;
        //printf("%lf %lf\n",creal(v[0]),creal(v[1]));  
        if (creal(v[0])==0){
        alpha=VectorNorm(v);
        }
        else{
        alpha=(result.R.mat[k][k]/cabs(result.R.mat[k][k]))*VectorNorm(v);
        }
        v[0]+=alpha;
        double complex vc=VectorNorm(v);
        if (vc!=0){
        for (int i=0;i<ORDER-k;i++){
            v[i]/=vc;
            }
        }
        double complex qprime[ORDER-k][ORDER-k];
        for(int i=0;i<ORDER-k;i++){
            for(int j=0;j<ORDER-k;j++){
                if (i==j){
                    qprime[j][i]=1-(2*conj(v[i])*v[j]);
                }
                else{
                    qprime[j][i]=-(2*conj(v[i])*v[j]);
                }
            }
        }
        

        for(int i=0;i<ORDER;i++){
            for (int j=0;j<ORDER;j++){
                if (i==j && i<k){H.mat[i][j]=1;}
                else if(i>=k&&j>=k){
                    H.mat[i][j]=qprime[i-k][j-k];
                }
                else{H.mat[i][j]=0;}
            }
        }
        result.R=MatMult(H,result.R);


        result.Q=MatMult(result.Q,trans(H));
    }


    return result;
}

double complex* QRAlgorithm(matrix A){
    double complex* eigenv = (double complex*)malloc(ORDER * sizeof(double complex));
    double complex shift;

    for (int  i=0; i< MAX_ITER; i++){
        if (tolcheck(A)==0){
            break;
        }
         shift = WilkinsonShift(A);
         matrix identity = Identity();
         A = MatSub(A, MatScalMult(identity, shift));
         QR qr = QRDecomposition(A);
         A = MatMult(qr.R,qr.Q);
        A=MatSub(MatMult(qr.R, qr.Q),MatScalMult(identity, -shift));       
    }
    
    int i;
    for (i=0;i<ORDER-1;i++){
        if (cabs(A.mat[i+1][i])>1e-5){
            double complex eig1,eig2,a=A.mat[i][i],b=A.mat[i][i+1],c=A.mat[i+1][i],d=A.mat[i+1][i+1];
            eigenv[i]=(a+d+csqrt((a+d)*(a+d)-4*(a*d-b*c)))/2;
            eigenv[i+1]=(a+d-csqrt((a+d)*(a+d)-4*(a*d-b*c)))/2;
            ++i;
            
        }
        else{
        eigenv[i]=A.mat[i][i];
        }
    }
    if (i==ORDER-1){
    eigenv[ORDER-1]=A.mat[ORDER-1][ORDER-1];}
    return eigenv;
}

int main(){
	double complex A[ORDER][ORDER] ={{1,2,5},
    {4,5,8},
    {2,4,6}};

    matrix Mat;

    for (int i=0;i<ORDER;i++){
        for (int j=0;j<ORDER;j++){
            Mat.mat[i][j]=A[i][j];
        }
    }                             
	
	double complex* eigen=(double complex*)malloc(ORDER*sizeof(double complex));
    eigen=QRAlgorithm(Mat);
	for (int i=0;i<ORDER;i++){
		printf("(%lf + %lfi)\n",creal(eigen[i]),cimag(eigen[i]));
	}
    
	return 0;
}
\end{verbatim}

\section{Properties of QR Algorithm with Householder technique and Wilkinson Shift}
\begin{itemize}
    \item Time complexity: $O\brak{n^3}$
    \item Wilkinson shift quadratically converges the approx eigenvalues i.e convergence rate is quadratic:
    $$\epsilon_k=\abs{\lambda_i-\lambda_{i,k}}$$
    $$\epsilon_{k+1}=C\epsilon_k^2$$
\end{itemize}

\section{Conclusion}
The QR decomposition and eigenvalue computation were successfully implemented using the Householder method. The results matched the expected theoretical outcomes.

\section{References}
\begin{itemize}
    \item Trefethen, L.N. and Bau, D. (1997) Numerical Linear Algebra. SIAM, Philadelphia. 
    \item Strang, Gilbert, Linear Algebra and Its Applications. New York, Academic Press, 1976. 
    \item Online resources: \url{https://www.wikipedia.org}
\end{itemize}

\centering{Thank You}
\end{document}
