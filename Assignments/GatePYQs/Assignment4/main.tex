%iffalse
\let\negmedspace\undefined
\let\negthickspace\undefined
\documentclass[journal,12pt,onecolumn]{IEEEtran}
\usepackage{cite}
\usepackage{amsmath,amssymb,amsfonts,amsthm}
\usepackage{algorithmic}
\usepackage{graphicx}
\usepackage{textcomp}
\usepackage{xcolor}
\usepackage{txfonts}
\usepackage{listings}
\usepackage{enumitem}
\usepackage{mathtools}
\usepackage{gensymb}
\usepackage{comment}
\usepackage[breaklinks=true]{hyperref}
\usepackage{tkz-euclide} 
\usepackage{listings}
\usepackage{gvv}                                        
%\def\inputGnumericTable{}                                 
\usepackage[latin1]{inputenc}     
\usepackage{xparse}
\usepackage{color}                                            
\usepackage{array}                                            
\usepackage{longtable}                                       
\usepackage{calc}                                             
\usepackage{multirow}
\usepackage{multicol}
\usepackage{hhline}                                           
\usepackage{ifthen}                                           
\usepackage{lscape}
\usepackage{tabularx}
\usepackage{array}
\usepackage{float}
\newtheorem{theorem}{Theorem}[section]
\newtheorem{problem}{Problem}
\newtheorem{proposition}{Proposition}[section]
\newtheorem{lemma}{Lemma}[section]
\newtheorem{corollary}[theorem]{Corollary}
\newtheorem{example}{Example}[section]
\newtheorem{definition}[problem]{Definition}
\newcommand{\BEQA}{\begin{eqnarray}}
\newcommand{\EEQA}{\end{eqnarray}}
\usepackage{float}
%\newcommand{\define}{\stackrel{\triangle}{=}}
\theoremstyle{remark}
\usepackage{ circuitikz }
%\newtheorem{rem}{Remark}
% Marks the beginning of the document
\begin{document}
\title{2013-CE-'53-65'}
\author{EE24BTECH11007 - Arnav Makarand Yadnopavit}
\maketitle
\renewcommand{\thefigure}{\theenumi}
\renewcommand{\thetable}{\theenumi}
%2011-PH-'53-65'
\parindent 0px A multistory building with a basement is to be constructed. The top 4$m$ consists of loose silt, below which dense sand layer is present up to a great depth. Ground water table is at the surface. The foundation consists of the basement slab of 6$m$ width which will rest on the top of dense sand as shown in the figure. For dense sand, saturated unit weight = 20$kN/m^3$, and bearing capacity factors $N_q = 40$ and $N_\gamma = 45$. For loose silt, saturated unit weight = 18$kN/m^3$, $N_q = 15$ and $N_\gamma = 20$. Effective cohesion $c\prime$ is zero for both soils. Unit weight of water is 10$kN/m^3$. Neglect shape factor and depth factor.\\
Average elastic modulus $E$ and Poisson's ratio $\mu$ of dense sand is $60\times10^3kN/m^2$ and 0.3 respectively.
\begin{figure}[H]
\centering
\begin{circuitikz}
\tikzstyle{every node}=[font=\LARGE]
\draw  (3.75,16) rectangle (13.75,9.75);
\draw [short] (3.75,14.75) -- (13.75,14.75);
\draw [short] (3.75,13.5) -- (13.75,13.5);
\draw [short] (3.75,12.25) -- (13.75,12.25);
\draw [short] (3.75,11) -- (13.75,11);
\draw [short] (6.25,16) -- (6.25,9.75);
\draw [short] (7.5,16) -- (9.5,16);
\draw [short] (8.75,16) -- (8.75,9.75);
\draw [short] (11.25,16) -- (11.25,9.75);
\draw  (6.25,14.75) circle (0.5cm);
\draw  (7.5,13.5) circle (0.4cm);
\draw  (6.25,11.5) circle (0.3cm);
\draw  (11.25,11) circle (0.2cm);
\node [font=\footnotesize, rotate around={90:(0,0)}] at (2.5,13.25) {(milligrams of microbe required to destroy half of the body mass in kilograms)};
\node [font=\footnotesize, rotate around={90:(0,0)}] at (2,13.25) {Toxicity};
\node [font=\footnotesize, rotate around={0:(0,0)}] at (8.75,9.25) {Potency};
\node [font=\footnotesize, rotate around={0:(0,0)}] at (8.75,9) {(Probability that microbe will overcome human immunity system)};
\node [font=\footnotesize] at (13.75,9.5) {1};
\node [font=\footnotesize] at (11.25,9.5) {0.8};
\node [font=\footnotesize] at (8.75,9.5) {0.6};
\node [font=\footnotesize] at (6.25,9.5) {0.4};
\node [font=\footnotesize] at (3.75,9.5) {0.2};
\node [font=\footnotesize, rotate around={0:(0,0)}] at (3.4,9.75) {0};
\node [font=\footnotesize, rotate around={0:(0,0)}] at (3.4,11) {200};
\node [font=\footnotesize, rotate around={0:(0,0)}] at (3.4,12.25) {400};
\node [font=\footnotesize, rotate around={0:(0,0)}] at (3.4,13.5) {600};
\node [font=\footnotesize, rotate around={0:(0,0)}] at (3.4,14.75) {800};
\node [font=\footnotesize, rotate around={0:(0,0)}] at (3.4,16) {1000};
\node [font=\footnotesize] at (7.25,15.5) {P(50mm)};
\node [font=\footnotesize] at (7.5,12.75) {Q(40mm)};
\node [font=\footnotesize] at (7.5,11.25) {R(30mm)};
\node [font=\footnotesize] at (12.5,11.5) {S(20mm)};
\end{circuitikz}

\end{figure}
\begin{enumerate}
\setcounter{enumi}{51}
\item Using factor of safety = 3, the net safe bearing capacity (in $kN/m^2$) of the foundation is:
\begin{enumerate}
\begin{multicols}{4}
\item 610
\item 320
\item 983
\item 693
\end{multicols}
\end{enumerate}
\item The foundation slab is subjected to vertical downward stresses equal to net safe bearing capacity derived in the above question. Using influence factor $I_f=2.0$, and neglecting embedment depth and rigidity corrections, the immediate settlement of the dense sand layer will be:
\begin{enumerate}
\begin{multicols}{4}
\item 58 $mm$
\item 111 $mm$
\item 126 $mm$
\item 179 $mm$
\end{multicols}
\end{enumerate}
\end{enumerate}
At a station, Storm I of 5 hour duration with intensity 2$cm/h$ resulted in a runoff of 4$cm$ and Storm II of 8 hour duration resulted in a runoff of 8.4$cm$. Assume that the $\phi$-index is the same for both the storms.
\begin{enumerate}
\setcounter{enumi}{53}
\item The $\phi$-index (in $cm/h$) is:
\begin{enumerate}
\begin{multicols}{4}
\item 1.2
\item 1.0
\item 1.6
\item 1.4
\end{multicols}
\end{enumerate}
\item  The Intensity of storm II (in $cm/h$) is:
\begin{enumerate}
\begin{multicols}{4}
\item 2.00
\item 1.75
\item 1.50
\item 2.25
\end{multicols}
\end{enumerate}
\item A number is as much greater than 75 as it is smaller than 117. The number is:
\begin{enumerate}
\begin{multicols}{4}
\item 91
\item 93
\item 89
\item 96
\end{multicols}
\end{enumerate}
\item \underline{The professor} \underline{ordered to} \underline{the students to go} \underline{out of the class}.\\
\hspace{3cm}I\hspace{3cm}II\hspace{2cm}III\hspace{3cm}IV\\
Which of the above underlined parts of the sentence is grammatically incorrect?
\begin{enumerate}
\begin{multicols}{4}
\item I
\item II
\item III
\item IV
\end{multicols}
\end{enumerate}
\item Which of the following options is the closest in meaning to the word given below:\\
Primeval
\begin{enumerate}
\begin{multicols}{2}
\item Modern
\item Historic
\item Primitive
\item Antique
\end{multicols}
\end{enumerate}
\item Friendship, no matter how \rule{3cm}{0.15mm} it is, has its limitations.
\begin{enumerate}
\item cordial
\item intimate
\item secret
\item pleasant
\end{enumerate}
\item Select the pair that best expresses a relationship similar to that expressed in the pair:\\
Medicine: Health
\begin{enumerate}
\item Science: Experiment
\item Wealth: Peace
\item Education: Knowledge
\item Money: Happiness
\end{enumerate}
\item $X$ and $Y$ are two positive real numbers such that $2X + Y \leq 6$ and $X + 2Y \leq 8$. For which of the following values  of $\brak{X,Y}$ the function $f\brak{X,Y} = 3X + 6Y$ will give the maximum value?
\begin{enumerate}
\item $\brak{\frac{4}{3},\frac{10}{3}}$
\item $\brak{\frac{8}{3},\frac{20}{3}}$
\item $\brak{\frac{8}{3},\frac{10}{3}}$
\item $\brak{\frac{4}{3},\frac{20}{3}}$
\end{enumerate}
\item If $\abs{4X-7} = 5$ then the values of $2\abs{X}-\abs{-X}$ is:
\begin{enumerate}
\begin{multicols}{4}
\item $2,\frac{1}{3}$
\item $\frac{1}{2}, 3$
\item $\frac{3}{2},9$
\item $\frac{2}{3},9$
\end{multicols}
\end{enumerate}
\item Following table provides figures (in rupees) on annual expenditure of a firm for two years - 2010
and 2011.
\begin{table}[h]
    \centering
    \begin{tabular}{|c|c|c|c|c|c|c|c|c|}
\hline
$t$ (hour) & 0 & 1 & 2 & 3 & 4 & 5 & 6 & 7 \\ \hline
$Q \brak{m^3/s}$ & 0 & 9 & 21 & 18 & 12 & 5 & 2 & 0 \\ \hline
\end{tabular}
\end{table}\\
In 2011, which of the following two categories have registered increase by same percentage?
\begin{enumerate}
\item Raw material and Salary \& wages
\item Salary \& wages and Advertising
\item Power \& fuel and Advertising
\item Raw material and Research \& Development
\end{enumerate}
\item A firm is selling its product at Rs. 60 per unit. The total cost of production is Rs. 100 and firm is
earning total profit of Rs. 500. Later, the total cost increased by 30\%. By what percentage the price
should be increased to maintained the same profit level.
\begin{enumerate}
\begin{multicols}{4}
\item 5
\item 10
\item 15
\item 30
\end{multicols}
\end{enumerate}
\item Abhishek is elder to Savar.\\
Savar is younger to Anshul.\\
Which of the given conclusions is logically valid and is inferred from the above
statements?
\begin{enumerate}
\item Abhishek is elder to Anshul
\item Anshul is elder to Abhishek
\item Abhishek and Anshul are of the same age
\item No conclusion follows
\end{enumerate}
\end{enumerate}
\end{document}
