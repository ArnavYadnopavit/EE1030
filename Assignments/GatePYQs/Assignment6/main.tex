%iffalse
\let\negmedspace\undefined
\let\negthickspace\undefined
\documentclass[journal,12pt,onecolumn]{IEEEtran}
\usepackage{cite}
\usepackage{amsmath,amssymb,amsfonts,amsthm}
\usepackage{algorithmic}
\usepackage{graphicx}
\usepackage{textcomp}
\usepackage{xcolor}
\usepackage{txfonts}
\usepackage{listings}
\usepackage{enumitem}
\usepackage{mathtools}
\usepackage{gensymb}
\usepackage{comment}
\usepackage[breaklinks=true]{hyperref}
\usepackage{tkz-euclide} 
\usepackage{listings}
\usepackage{gvv}                                        
%\def\inputGnumericTable{}                                 
\usepackage[latin1]{inputenc}     
\usepackage{xparse}
\usepackage{color}                                            
\usepackage{array}                                            
\usepackage{longtable}                                       
\usepackage{calc}                                             
\usepackage{multirow}
\usepackage{multicol}
\usepackage{hhline}                                           
\usepackage{ifthen}                                           
\usepackage{lscape}
\usepackage{tabularx}
\usepackage{array}
\usepackage{float}
\newtheorem{theorem}{Theorem}[section]
\newtheorem{problem}{Problem}
\newtheorem{proposition}{Proposition}[section]
\newtheorem{lemma}{Lemma}[section]
\newtheorem{corollary}[theorem]{Corollary}
\newtheorem{example}{Example}[section]
\newtheorem{definition}[problem]{Definition}
\newcommand{\BEQA}{\begin{eqnarray}}
\newcommand{\EEQA}{\end{eqnarray}}
\usepackage{float}
%\newcommand{\define}{\stackrel{\triangle}{=}}
\theoremstyle{remark}
\usepackage{ circuitikz }
%\newtheorem{rem}{Remark}
% Marks the beginning of the document
\begin{document}
\title{2021-MA-'40-52'}
\author{EE24BTECH11007 - Arnav Makarand Yadnopavit}
\maketitle
\renewcommand{\thefigure}{\theenumi}
\renewcommand{\thetable}{\theenumi}
%2021-MA-'40-52'
\begin{enumerate}
\setcounter{enumi}{39}
\item The equation $xy-z\log{y}+e^{xz} = 1$ can be solved in a neighborhood of the point $\brak{0, 1, 1}$ as $y = f\brak{x, z}$ for some continuously differentiable function $f$. Then
\begin{enumerate}
\begin{multicols}{2}
\item $\nabla f\brak{0, 1} = \brak{2, 0}$
\item $\nabla f\brak{0, 1} = \brak{0, 2}$
\item $\nabla f\brak{0, 1} = \brak{0, 1}$
\item $\nabla f\brak{0, 1} = \brak{1, 0}$
\end{multicols}
\end{enumerate}
\item Consider the following topologies on the set $\mathbb{R}$ of all real numbers.\\
$T_1$ is the upper limit topology having all sets $\brak{a, b}$ as basis.\\
$T_2=\cbrak{U\subset \mathbb{R}: \mathbb{R}\backslash U \text{is finite}}\cup \cbrak{\phi}$.\\
$T_3$ is the standard topology having all sets $\brak{a, b}$ as basis. Then
\begin{enumerate}
\begin{multicols}{4}
\item $T_2\subset T_3\subset T_1$
\item $T_1\subset T_2\subset T_3$
\item $T_3\subset T_2\subset T_1$
\item $T_2\subset T_1\subset T_3$
\end{multicols}
\end{enumerate}
\item Let $\mathbb{R}$ denote the set of all real numbers. Consider the following topological spaces.\\
$X_1 = \brak{\mathbb{R}, T_1}$, where $T_1$ is the upper limit topology having all sets $\brak{a, b}$ as basis.\\
$X_2=\brak{\mathbb{R}, T_2}$, where $T_2=\cbrak{U\subset \mathbb{R}: \mathbb{R}\backslash U \text{is finite}} \cup \cbrak{\phi}$. Then
\begin{enumerate}
\item both $X_1$ and $X_2$ are connected
\item $X_1$ is connected and $X_2$ is NOT connected
\item $X_1$ is NOT connected and $X_2$ is connected
\item neither $X_1$ nor $X_2$ is connected
\end{enumerate}
\item Let $\langle \cdot,\cdot\rangle: \mathbb{R}^n \times \mathbb{R}^n\rightarrow\mathbb{R}$ be an inner product on the vector space $\mathbb{R}$ over $\mathbb{R}$. Consider the following statements:\\
$P: \abs{\langle u, v\rangle} \leq \frac{1}{2}\brak{\langle u, u\rangle + \langle v, v\rangle}$ for all $u, v \in \mathbb{R}^n$.\\
$Q: If \langle u, v\rangle = \langle 2u, - v\rangle$ for all $v\in\mathbb{R}^n$, then $u = 0$. Then
\begin{enumerate}
\item both P and Q are TRUE
\item P is TRUE and Q is FALSE
\item P is FALSE and Q is TRUE
\item both P and Q are FALSE
\end{enumerate}
\item Let $G$ be a group of order $5^4$ with center having $5^2$ elements. Then the number of conjugacy classes in $G$ is \rule{2cm}{0.15mm}.
\item Let $F$ be a finite field and $F^\times$ be the group of all nonzero elements of $F$ under multiplication. If $F^\times$ has a subgroup of order 17, then the smallest possible order of the field $F$ is \rule{2cm}{0.15mm}
\item Let $R= \cbrak {z = x + iy \in \mathbb{C} : 0 < x < 1 \text{and} - 11\pi < y < 11\pi }$ and $\Gamma$ be the positively oriented boundary of $R$. Then the value of the integral
\[
\frac{1}{2\pi i}\int\limits_\Gamma\frac{e^z dz}{e^z -2}
\]
is \rule{2cm}{0.15mm}.
\item Let $D = \cbrak {z \in \mathbb{C} : \abs{z}<2\pi}$ and $f:D\rightarrow \mathbb{C}$ be the function defined by\\
$$f\brak{z} = \begin{cases}
\frac{3z^2}{\brak{1-\cos z}} & \text{if } z \neq 0, \\
6 & \text{if } z = 0.
\end{cases}$$
If $f\brak{z} = \sum^\infty_{n=0}a_nz^n$ for $z \in D$, then $6a_2=$\rule{2cm}{0.15mm}
\item The number of zeroes (counting multiplicity) of $P\brak{z} = 3z^5+2iz^2+7iz+1$ in the annular region $\cbrak{z\in \mathbb{C}:1<\abs{z}<7}$ is \rule{2cm}{0.15mm}.
\item Let $A$ be a square matrix such that det$\brak{xI-A}=x^4\brak{x-1}^2\brak{x-2}^3$, where det$\brak{M}$ denotes the determinant of a square matrix $M$.\\
If rank$\brak{A^2}<\brak{A^3}=\brak{A^4}$, then the geometric multiplicity of the eigenvalue 0 of A is \rule{2cm}{0.15mm}.
\item If $y = \sum^\infty_{k=0}a_kx^k$, $\brak{a_0\neq0}$ is the power series solution of the differential equation $\frac{d^2y}{dx^2}-24x^2y=0$, the $\frac{a_4}{a_0}=$ \rule{2cm}{0.15mm}.
\item If $u\brak{x,t}=Ae^{-t}\sin x$ solves the following initial boundary value problem\\
$\begin{aligned}
    \frac{\partial u}{\partial t} = \frac{\partial^2 u}{\partial x^2}, \quad 0 < x < \pi, \quad t > 0, \\
    u\brak{0,t} = u\brak{\pi,t} = 0, \quad t > 0, \\
    u\brak{x,0} = \begin{cases}
        60,  0 < x \leq \frac{\pi}{2}, \\
        40,  \frac{\pi}{2} < x < \pi,
    \end{cases}
\end{aligned}$\\
then $\pi A=$\rule{2cm}{0.15mm}
\item Let $V = \cbrak{p: p\brak{x} = a_0 + a_1x + a_2x^2, a_0, a_1, a_2 \in \mathbb{R}}$ be the vector space of all polynomials of degree at most 2 over the real field $\mathbb{R}$. Let $T: V \rightarrow V$ be the linear operator given by

\[T\brak{p} = \brak{p\brak{0} - p\brak{1}} + \brak{p\brak{0} + p\brak{1}}x + p\brak{0}x^2.\]

Then the sum of the eigenvalues of $T$ is \rule{2cm}{0.15mm}.
\end{enumerate}
\end{document}
