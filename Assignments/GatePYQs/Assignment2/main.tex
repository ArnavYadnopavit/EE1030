%iffalse
\let\negmedspace\undefined
\let\negthickspace\undefined
\documentclass[journal,12pt,onecolumn]{IEEEtran}
\usepackage{cite}
\usepackage{amsmath,amssymb,amsfonts,amsthm}
\usepackage{algorithmic}
\usepackage{graphicx}
\usepackage{textcomp}
\usepackage{xcolor}
\usepackage{txfonts}
\usepackage{listings}
\usepackage{enumitem}
\usepackage{mathtools}
\usepackage{gensymb}
\usepackage{comment}
\usepackage[breaklinks=true]{hyperref}
\usepackage{tkz-euclide} 
\usepackage{listings}
\usepackage{gvv}                                        
%\def\inputGnumericTable{}                                 
\usepackage[latin1]{inputenc}     
\usepackage{xparse}
\usepackage{color}                                            
\usepackage{array}                                            
\usepackage{longtable}                                       
\usepackage{calc}                                             
\usepackage{multirow}
\usepackage{multicol}
\usepackage{hhline}                                           
\usepackage{ifthen}                                           
\usepackage{lscape}
\usepackage{tabularx}
\usepackage{array}
\usepackage{float}
\newtheorem{theorem}{Theorem}[section]
\newtheorem{problem}{Problem}
\newtheorem{proposition}{Proposition}[section]
\newtheorem{lemma}{Lemma}[section]
\newtheorem{corollary}[theorem]{Corollary}
\newtheorem{example}{Example}[section]
\newtheorem{definition}[problem]{Definition}
\newcommand{\BEQA}{\begin{eqnarray}}
\newcommand{\EEQA}{\end{eqnarray}}
%\newcommand{\define}{\stackrel{\triangle}{=}}
\theoremstyle{remark}
%\newtheorem{rem}{Remark}
% Marks the beginning of the document
\begin{document}
\title{Assignment2}
\author{EE24BTECH11007 - Arnav Makarand Yadnopavit}
\maketitle
\renewcommand{\thefigure}{\theenumi}
\renewcommand{\thetable}{\theenumi}
\begin{enumerate}
\item For a flow through a Prandtl-Meyer expansion wave:
\begin{enumerate}
\begin{multicols}{2}
\item Mach number stays constant.
\item Entropy stays constant.
\item Temperature stays constant.
\item Density stays constant.
\end{multicols}
\end{enumerate}
\item For two-dimensional irrotational and incompressible flows:
\begin{enumerate}
\item Both potential and stream functions satisfy the Laplace equation.
\item Potential function must satisfy the Laplace equation but the stream function need not.
\item Stream function must satisfy the Laplace equation but the potential function need not.
\item Neither the stream function nor the potential function need to satisfy the Laplace equation.
\end{enumerate}
\item A trailing edge plain flap deflected downward increases the lift coefficient of an airfoil by
\begin{enumerate}
\item Increasing the effective camber of the airfoil.
\item Delaying the separation of the flow from the airfoil surface.
\item Increasing the local airspeed near the trailing edge.
\item Controlling the growth of the boundary layer thickness along the airfoil surface.
\end{enumerate}
\item Thin airfoil theory predicts that the lift slope is $\frac{dc_l}{d\alpha}=2\pi$ for:
\begin{enumerate}
\begin{multicols}{2}
\item Symmetric airfoils only.
\item Cambered airfoils only.
\item Any airfoil shape.
\item Joukowski airfoils only.
\end{multicols}
\end{enumerate}
\item The ordinary differential equation $\frac{d^2y}{dx^2}+ky=0$, where $k$ is real and positive:
\begin{enumerate}
\item is non-linear.
\item has a characteristic equation with one real and one complex root.
\item has a characteristic equation with two real roots.
\item has a complementary function that is simple harmonic.
\end{enumerate}
\item A non-trivial solution to the $\brak{n\times n}$ system of equations $\sbrak{A}\cbrak{x} = \cbrak{0}$, where $\cbrak{0}$ is the null vector:
\begin{enumerate}
\item can never be found.
\item may be found only if $\sbrak{A}$ is not singular.
\item may be found only if $\sbrak{A}$ is an orthogonal matrix.
\item may be found only if $\sbrak{A}$ has at least one eigenvalue equal to zero.
\end{enumerate}
\item For a plane strain problem, the stresses satisfy the condition:
\begin{enumerate}
\item $\tau_{xz}=\tau_{yz}=\sigma_z=0$
\item $\tau_{xz}=\tau_{yz}=0,\sigma_z=\nu\brak{\sigma_x+\sigma_y}$
\item $\tau_{xz}=\tau_{yz}=0.,\sigma_z=\nu\tau_{xy}$
\item $\tau_{xz}=\tau_{yz}=0,\sigma_z=\nu\brak{\sigma_x+\sigma_y}+\brak{1-\nu}\tau_{xy}$
\end{enumerate}
\item The propulsive efficiency of a turbo-jet engine moving at velocity $U_\infty$ and having exhaust velocity $U_e$ with respect to the engine is given by:
\begin{enumerate}
\begin{multicols}{4}
\item $\frac{2}{U_\infty/U_e+1}$
\item $1-\frac{U_\infty}{U_e}$
\item  $\frac{2U_\infty U_e}{U_e^2+U_\infty^2}$
\item $\frac{2U_\infty}{U_e+U_\infty}$
\end{multicols}
\end{enumerate}
\item An aircraft is flying at $M = 2$ where the ambient temperature around the aircraft is 250$K$. If the specific heat ratio for air $\gamma$ = 1.4, the stagnation temperature on the surface of the aircraft is:
\begin{enumerate}
\begin{multicols}{4}
\item 200$K$
\item 450$K$
\item 350$K$
\item 1450$K$
\end{multicols}
\end{enumerate}
\item The division of feed air to an aircraft gas-turbine combustor into primary and secondary streams serves which of the following purposes?\\
P. A flammable mixture can be formed\\
Q. Cooling of combustor liner and flame tube can be accomplished\\
R. Specific fuel consumption can be reduced
\begin{enumerate}
\begin{multicols}{4}
\item P and R
\item Q ad R
\item P and Q
\item P,Q and R
\end{multicols}
\end{enumerate}
\item Classify the following propellants as: cryogenic \brak{C}, semi-cryogenic \brak{SC}, compressed gas \brak{CG}, and earth storable \brak{ES}.\\
N$_2$O$_4$-UDMH \brak{nitrogen tetra oxide and unsymmetrical di-methyl hydrazine}\\
LOX-RP1 \brak{liquid oxygen and kerosene}\\
LOX-LH$_2$ \brak{liquid oxygen and liquid hydrogen}\\
N$_2$ \brak{nitrogen gas}\\
\begin{enumerate}
\item N$_2$O$_4$-UDMH \brak{ES}, LOX-RP1 \brak{C}, LOX-LH$_2$ \brak{C}, N$_2$ \brak{C}
\item N$_2$O$_4$-UDMH \brak{SC}, LOX-RP1 \brak{SC}, LOX-LH$_2$ \brak{C}, N$_2$ \brak{C}
\item N$_2$O$_4$-UDMH \brak{ES}, LOX-RP1 \brak{SC}, LOX-LH$_2$ \brak{C}, N$_2$ \brak{CG}
\item N$_2$O$_4$-UDMH \brak{ES}, LOX-RP1 \brak{C}, LOX-LH$_2$ \brak{C}, N$_2$ \brak{CG}
\end{enumerate}
\item A conventional altimeter is a:
\begin{enumerate}
\begin{multicols}{2}
\item Pressure transducer
\item Temperature transducer
\item Density transducer
\item Velocity transducer
\end{multicols}
\end{enumerate}
\end{enumerate}
\end{document}
