%iffalse
\let\negmedspace\undefined
\let\negthickspace\undefined
\documentclass[journal,12pt,onecolumn]{IEEEtran}
\usepackage{cite}
\usepackage{amsmath,amssymb,amsfonts,amsthm}
\usepackage{algorithmic}
\usepackage{graphicx}
\usepackage{textcomp}
\usepackage{xcolor}
\usepackage{txfonts}
\usepackage{listings}
\usepackage{enumitem}
\usepackage{mathtools}
\usepackage{gensymb}
\usepackage{comment}
\usepackage[breaklinks=true]{hyperref}
\usepackage{tkz-euclide} 
\usepackage{listings}
\usepackage{gvv}                                        
%\def\inputGnumericTable{}                                 
\usepackage[latin1]{inputenc}     
\usepackage{xparse}
\usepackage{color}                                            
\usepackage{array}                                            
\usepackage{longtable}                                       
\usepackage{calc}                                             
\usepackage{multirow}
\usepackage{multicol}
\usepackage{hhline}                                           
\usepackage{ifthen}                                           
\usepackage{lscape}
\usepackage{tabularx}
\usepackage{array}
\usepackage{float}
\newtheorem{theorem}{Theorem}[section]
\newtheorem{problem}{Problem}
\newtheorem{proposition}{Proposition}[section]
\newtheorem{lemma}{Lemma}[section]
\newtheorem{corollary}[theorem]{Corollary}
\newtheorem{example}{Example}[section]
\newtheorem{definition}[problem]{Definition}
\newcommand{\BEQA}{\begin{eqnarray}}
\newcommand{\EEQA}{\end{eqnarray}}
%\newcommand{\define}{\stackrel{\triangle}{=}}
\theoremstyle{remark}
%\newtheorem{rem}{Remark}
% Marks the beginning of the document
\begin{document}
\title{Assignment1}
\author{EE24BTECH11007 - Arnav Makarand Yadnopavit}
\maketitle
\renewcommand{\thefigure}{\theenumi}
\renewcommand{\thetable}{\theenumi}
\begin{enumerate}
\item A triangular open channel has a vertex angle of $90\degree$ and carries flow at a critical depth of 0.30 m. The discharge in the channel is
\begin{enumerate}
\begin{multicols}{4}
\item 0.08 m$^3$ /s
\item 0.11 m$^3$ /s
\item 0.15 m$^3$ /s
\item 0.2 m$^3$ /s
\end{multicols}
\end{enumerate}
\item Flow rate of a fluid \brak{\text{density}= 1000 kg/m^3} in a small diameter tube is 800 mm$^3$/s. The length and the diameter of the tube are 2 m and 0.5 mm, respectively. The pressure drop in 2 m length is equal to 2.0 MPa. The viscosity of the fluid is 
\begin{enumerate}
\begin{multicols}{4}
\item 0.025 N.s/m$^2$
\item 0.012 N.s/m$^2$
\item 0.00192 N.s/m$^2$
\item 0.00102 N.s/m$^2$
\end{multicols}
\end{enumerate}
\item The flow rate in a wide rectangular open channel is 2.0 m$^3$/s per meter width. The channel bed slope is 0.002. The Manning's roughness coefficient is 0.012. The slope of the channel is classified as
\begin{enumerate}
\begin{multicols}{4}
\item Critical
\item Horizontal
\item Mild
\item Steep
\end{multicols}
\end{enumerate}
\item The culturable command area for a distributary channel is 20,000 hectares. Wheat is grown in the entire area and the intensity of irrigation is 50\%. The kor period for wheat is 30 days and the kor water depth is 120 mm. The outlet discharge for the distributary should be
\begin{enumerate}
\begin{multicols}{4}
\item 2.85 m$^3$/s
\item 3.21 m$^3$/s
\item 4.63 m$^3$/s
\item 5.23 m$^3$/s
\end{multicols}
\end{enumerate}
\item An isolated 4-hour storm occurred over a catchment as follows\\
\begin{table}[h]
    \centering
    \begin{tabular}{|c|c|c|c|c|c|c|c|c|}
\hline
$t$ (hour) & 0 & 1 & 2 & 3 & 4 & 5 & 6 & 7 \\ \hline
$Q \brak{m^3/s}$ & 0 & 9 & 21 & 18 & 12 & 5 & 2 & 0 \\ \hline
\end{tabular}
\end{table}\\
The $\phi$ index for the catchment is 10 mm/h. The estimated runoff depth from the catchment due to the above storm is
\begin{enumerate}
\begin{multicols}{4}
\item 10 mm
\item 16 mm
\item 20 mm
\item 23 mm
\end{multicols}
\end{enumerate}
\item Two electrostatic precipitators \brak{\text{ESPs}} are in series. The fractional efficiencies of the upstream and downstream ESPs for size d$_p$ are 80\% and 65\%, respectively. What is the overall efficiency of the system for the same d$_p$?
\begin{enumerate}
\begin{multicols}{4}
\item 100\%
\item 93\%
\item 80\%
\item 65\%
\end{multicols}
\end{enumerate}
\item 50 g of CO$_2$ and 25 g of CH$_4$ are produced from the decomposition of municipal solid waste (MSW) with a formula weight of 120 g. What is the average per capita green house gas production in a city of 1 million people with a MSW production rate of 500 ton/day?
\begin{enumerate}
\begin{multicols}{4}
\item 104 g/day
\item 120 g/day
\item 208 g/day
\item 313 g/day
\end{multicols}
\end{enumerate}
\item The extra widening required for a two-lane national highway at a horizontal curve of 300 m radius, considering a wheel base of 8 m and a design speed of 100 kmph is
\begin{enumerate}
\begin{multicols}{4}
\item 0.42 m
\item 0.62 m
\item  0.82 m
\item 0.92 m
\end{multicols}
\end{enumerate}
\item While designing a hill road with a ruling gradient of 6\%, if a sharp horizontal curve of 50 m radius is encountered, the compensated gradient at the curve as per the Indian Roads Congress specifications should be
\begin{enumerate}
\begin{multicols}{4}
\item 4.4\%
\item 4.75\%
\item 5.0\%
\item 5.25\%
\end{multicols}
\end{enumerate}
\item The design speed on a road is 60 kmph. Assuming the driver reaction time of 2.5 seconds and coefficient of friction of pavement surface as 0.35, the required stopping distance for two-way traffic on a single lane road is
\begin{enumerate}
\begin{multicols}{4}
\item 82.1 m
\item 102.4 m
\item 164.2 m
\item 186.4
\end{multicols}
\end{enumerate}
\item The width of the expansion joint is 20 mm in a cement concrete pavement. The laying temperature is 20\degree C and the maximum slab temperature in summer is 60\degree C. The coefficient of thermal expansion of concrete is $10\times10^{-6}$ mm/mm/\degree C and the joint filler compresses up to 50\% of the thickness. The spacing between expansion joints should be 
\begin{enumerate}
\begin{multicols}{4}
\item 20m
\item 25m
\item 30m
\item 40m
\end{multicols}
\end{enumerate}
\item The following data pertains to the number of commercial vehicles per day for the design of a flexible pavement for a national highway as per IRC:37-1984:\\
\begin{table}[h]
    \centering
    \begin{tabular}{|c|p{5cm}|p{3cm}|}
    \hline
    Type of commercial vehicle & Number of vehicles per day-considering the number of lanes & Vehicle Damage Factor\\
    \hline
    Two axle trucks & 2000 & 5\\
    \hline
    Tandem axle trucks & 200 & 6\\
    \hline
\end{tabular}
\end{table}\\
Assuming a traffic growth factor of 7.5\% per annum for both the types of vehicles, the cumulative number of standard axle load repetitions \brak{\text{in million}} for a design life of ten years is
\begin{enumerate}
\begin{multicols}{4}
\item 44.6
\item 57.8
\item 62.4
\item 78.7
\end{multicols}
\end{enumerate}
\item Match the following tests on aggregate and its properties.
\begin{multicols}{2}
TEST\\
P. Crushing Test\\
Q. Los Angeles abrasion test\\
R. Soundness test\\
S. Angularity test\\
PROPERTY\\
1. Hardness\\
2. Weathering\\
3. Shape\\
4. Strength
\end{multicols}
\begin{enumerate}
\begin{multicols}{4}
\item P-2, Q-1, R-4, S-3
\item P-4, Q-2, R-3, S-1
\item P-3, Q-2, R-1, S-4
\item P-4, Q-1, R-2,S-3
\end{multicols}
\end{enumerate}
\item The plan of a map was photocopied to a reduced size such that a line originally 100 mm, measures 90 mm. The original scale of the plan was 1:1000. The revised scale is
\begin{enumerate}
\begin{multicols}{4}
\item 1:900
\item 1:1111
\item 1:1121
\item 1:1221
\end{multicols}
\end{enumerate}
\item The following table gives data of consecutive coordinates in respect of a closed theodolite traverse PQRSP. \\
\begin{table}[h]
    \centering
    \begin{tabular}{|c|p{1.5cm}|p{1.5cm}|p{1.5cm}|p{1.5cm}|}
    \hline
    Station & Northing,$m$ & Southing,$m$ & Easting,$m$ & Westing,$m$\\
    \hline
    P & 400.75 &  &  & 300.5 \\
    \hline
    Q & 100.25 &  & 199.25 &  \\
    \hline
    R &  & 199.0 & 399.75 &  \\
    \hline
    S &  & 300.0 &  & 200.5 \\
    \hline
\end{tabular}
\end{table}\\
\begin{enumerate}
\begin{multicols}{4}
\item 2.0 m and 45\degree
\item 2.0 m and 315\degree
\item 2.82 m and 315\degree
\item 3.42 m and 45\degree
\end{multicols}
\end{enumerate}
\item The following measurements were made during testing a leveling instrument.
\begin{table}[h]
    \centering
    \begin{tabular}{|c|p{3cm}|p{3cm}|}
    \hline
    \multirow{2}{*}{Instrument at} & \multicolumn{2}{|c|}{Staff Reading at}\\
    \cline{2-3}
    & P$_1$ & Q$_1$ \\
    \hline
    P & 2.800$m$ & 1.700$m$\\
    \hline
    Q & 2.700$m$ & 1.800$m$\\
    \hline
\end{tabular}
\end{table}\\
P$_1$ is close to P and Q$_1$ is close to Q. If the reduced level of station P is 100.000 m, the reduced level of station Q is
\begin{enumerate}
\begin{multicols}{4}
\item 99.000 m
\item 100.000 m
\item 101.000 m
\item 102.000 m
\end{multicols}
\end{enumerate}
\item Two straight lines intersect at an angle of 60\degree. The radius of a curve joining the two straight lines is 600 m. The length of long chord and mid-ordinates in meters of the curve are
\begin{enumerate}
\begin{multicols}{4}
\item 80.4, 600.0
\item 600.0, 80.4
\item 600.0, 39.89
\item 49.89, 300.0
\end{multicols}
\end{enumerate}
\end{enumerate}
\end{document}