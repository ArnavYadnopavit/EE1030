%iffalse
\let\negmedspace\undefined
\let\negthickspace\undefined
\documentclass[journal,12pt,twocolumn]{IEEEtran}
\usepackage{cite}
\usepackage{amsmath,amssymb,amsfonts,amsthm}
\usepackage{algorithmic}
\usepackage{graphicx}
\usepackage{textcomp}
\usepackage{xcolor}
\usepackage{txfonts}
\usepackage{listings}
\usepackage{enumitem}
\usepackage{mathtools}
\usepackage{gensymb}
\usepackage{comment}
\usepackage[breaklinks=true]{hyperref}
\usepackage{tkz-euclide} 
\usepackage{listings}
\usepackage{gvv}                                        
%\def\inputGnumericTable{}                                 
\usepackage[latin1]{inputenc}                                
\usepackage{color}                                            
\usepackage{array}                                            
\usepackage{longtable}                                       
\usepackage{calc}                                             
\usepackage{multirow}                                         
\usepackage{hhline}                                           
\usepackage{ifthen}                                           
\usepackage{lscape}
\usepackage{tabularx}
\usepackage{array}
\usepackage{float}
\usepackage{multicol}
\newcounter {sectioicolsn}

\newtheorem{theorem}{Theorem}[section]
\newtheorem{problem}{Problem}
\newtheorem{proposition}{Proposition}[section]
\newtheorem{lemma}{Lemma}[section]
\newtheorem{corollary}[theorem]{Corollary}
\newtheorem{example}{Example}[section]
\newtheorem{definition}[problem]{Definition}
\newcommand{\BEQA}{\begin{eqnarray}}
\newcommand{\EEQA}{\end{eqnarray}}
\newcommand{\define}{\stackrel{\triangle}{=}}
\theoremstyle{remark}
\newtheorem{rem}{Remark}
% Marks the beginning of the document
\begin{document}
\bibliographystyle{IEEEtran}
\vspace{3cm}

\title{EE1030: Matrix Theory}
\author{EE24BTECH11007 - Arnav Makarand Yadnopavit}
\maketitle
\newpage
\bigskip

\renewcommand{\thefigure}{\theenumi}
\renewcommand{\thetable}{\theenumi}

\begin{enumerate}[start=6,label={\arabic*}.]
\item The number of distinct solutions of equation
\medskip
$\frac{5}{4}cos^2 2x+cos^4 x+sin^4 x+cos^6 x+sin^6 x=2$
\smallskip

in the interval [0,2$\pi$] is\hfill{$($JEE Adv. $2015)$} 
%$($JEE Adv. $2015)$
\medskip
    \item Let $a, b, c$ be three non-zero real numbers such\smallskip
    
    that the equation:
    \smallskip

$\sqrt{3} acosx+2bsinx = c,x\in \left[\frac{-\pi}{2},\frac{\pi}{2}\right]$
, has two distinct real roots $\alpha$ and $\beta$ with $\alpha+\beta=\frac{\pi}{3}$. Then, the value of $\frac{b}{a}$ is \underline{\hspace{2cm}}.\hspace{0.5cm}

\hfill{$($JEE Adv. $2018)$}
\end{enumerate}
\bigskip

{\LARGE Section-B JEE Main/AIEEE}
\bigskip
%Use this
%$(a)$ $o$\hspace{1cm}  $(b)$ $o$\hspace{1cm} $(c)$ $o$\hspace{1cm} $(d)$ $o$
 %or
 %$(a)$ $o$\hspace{2cm} $(b)$ $o$

\begin{enumerate}[label={\arabic*}.]
\item The period of $sin^2 \theta$ is\hfill{$[2002]$} 

\begin{multicols}{4}

\brak{a} $\pi^2$
\columnbreak

\brak{b} $\pi$
\columnbreak

\brak{c} $2\pi$
\columnbreak

\brak{d} $\pi/2$
\end{multicols}
\medskip

\item The number of solution of $tanx + secx=2cosx$ in $[0,2\pi]$ is\hfill{$[2002]$} 

\begin{multicols}{4}

\brak{a} $2$
\columnbreak

\brak{b} $3$
\columnbreak

\brak{c} $0$
\columnbreak

\brak{d} $1$
\end{multicols}
\medskip

\item Which one is not periodic \hfill{$[2002]$}
\medskip
\begin{multicols}{2}

\brak{a} $|sin3x|+sin^2 x$

\brak{c} $cos4x+tan^2 x$
\columnbreak

\brak{b} $cos\sqrt{x}+cos^2 x$

\brak{d} $cos2x+sinx$
\end{multicols}

\medskip
\item Let $\alpha,\beta$ be such that $\pi<\alpha-\beta<3\pi$

\smallskip
If $sin\alpha+sin\beta=-\frac{21}{65}$ and $cos\alpha+cos\beta=-\frac{27}{65}$, then the value of $cos\frac{\alpha-\beta}{2}$\hfill{$[2004]$}
\medskip

\begin{multicols}{2}

\brak{a} $-\frac{6}{65}$

\brak{c} $\frac{6}{65}$
\columnbreak

\brak{b} $\frac{3}{\sqrt{130}}$

\brak{d} $-\frac{3}{\sqrt{130}}$

\end{multicols}

\medskip
\item If $u=\sqrt{a^2 cos^2 \theta+b^2 sin^2 \theta}$$+$$\sqrt{a^2 sin^2 \theta+b^2 cos^2 \theta}$
then the difference between the  maximum and minimum values of $u^2$ is given by \hfill{$[2004]$}
\medskip
\begin{multicols}{2}

\brak{a} $(a-b)^2$

\brak{c} $(a+b)^2$
\columnbreak

\brak{b} $2\sqrt{a^2 +b^2}$

\brak{d} $2(a^2 +b^2)$

\end{multicols}

\medskip
 
\item A line makes the same angle $\theta$, with each of the x and z axis. 
If the angle $\beta$, which it makes with y-axis, is such that
$sin^2 \beta=3sin^2 \theta$, then $cos^2 \theta$ equals \hfill{$[2004]$}
\medskip

\begin{multicols}{2}

\brak{a} $\frac{2}{5}$
\medskip

\brak{c} $\frac{3}{5}$
\columnbreak

\brak{b} $\frac{1}{5}$
\medskip

\brak{d} $\frac{2}{3}$

\end{multicols}

\medskip

\item The number of values of $x$ in the interval $[0,3\pi]$ satisfying the equation 

$2sin^2 x+5sinx-3=0$ is \hfill{$[2006]$}

\medskip
\begin{multicols}{4}

\brak{a} 4
\columnbreak

\brak{b} 6
\columnbreak

\brak{c} 1
\columnbreak

\brak{d} 2

\end{multicols}
\medskip
\item If $0<x<\pi$ and $cosx+sinx=\frac{1}{2}$, then $tanx$ is 

\hfill{$[2006]$}
\begin{multicols}{2}

\brak{a} $\frac{(1-\sqrt{7})}{4}$

\brak{c} $-\frac{(4+\sqrt{7}}{3}$
\columnbreak

\brak{b} $\frac{(4-\sqrt{7})}{3}$

\brak{d} $\frac{(1+\sqrt{7})}{4}$

\end{multicols}

\medskip

\item Let \textbf{A} and \textbf{B} denote the statements

\textbf{A}:$cos\alpha+cos\beta+cos\gamma=0$

\textbf{B}:$sin\alpha+sin\beta+sin\gamma=0$

If $cos(\beta-\gamma)+cos(\gamma-\alpha)+cos(\alpha-\beta)=-\frac{3}{2}$,then:

\smallskip
\hfill{$[2009]$}

\medskip
\brak{a} \textbf{A} is false and \textbf{B} is true 

\brak{b} both \textbf{A} and \textbf{B} are true

\brak{c} both \textbf{A} and \textbf{B} are false 

\brak{d} \textbf{A} is true and \textbf{B} is false

\medskip
\item Let $cos(\alpha+\beta)=\frac{4}{5}$  and $sin(\alpha-\beta)=\frac{5}{13}$, where $0\le\alpha$, $\beta\le\frac{\pi}{4}$. Then $tan2\alpha=$ \hfill{$[2010]$}
\begin{multicols}{4}

\brak{a} $\frac{56}{33}$
\columnbreak

\brak{b} $\frac{19}{12}$
\columnbreak

\brak{c} $\frac{20}{7}$
\columnbreak

\brak{d} $\frac{25}{16}$

\end{multicols}

\item If A=$sin^2x +cos^4 x$, Then for all real $x$:

\hfill{$[2010]$}
\medskip

\begin{multicols}{2}

\brak{a} $\frac{13}{16}\le$A$\le1$

\brak{c} $\frac{3}{4}\le$A$\le\frac{13}{16}$
\columnbreak

\brak{b} $1\le$A$\le2$

\brak{d} $\frac{3}{4}\le$A$\le1$
\end{multicols}

\medskip

\item In a $\Delta$PQR, If $3 sinP + 4 cosQ=6$ and $4sinQ+3cosP=1$, then the angle $R$ is equal to: 

\hfill{$[2012]$}
\medskip
\begin{multicols}{4}

\brak{a} $\frac{5\pi}{6}$
\columnbreak

\brak{b} $\frac{\pi}{6}$
\columnbreak

\brak{c} $\frac{\pi}{4}$
\columnbreak

\brak{d} $\frac{3\pi}{4}$

\end{multicols}

\item ABCD is a trapezium such that AB and CD are parallel and BC$\perp$CD. If $\angle$ADB=$\theta$, BC=$p$ and CD=$q$, then AB is equal to:

\hfill{$[$JEEM2013$]$}
\medskip
\begin{multicols}{2}

\brak{a} $\frac{(p^2+q^2)sin\theta}{pcos\theta+qsin\theta}$

\brak{c} $\frac{p^2+q^2}{pcos^2 \theta+qsin^2 \theta}$
\columnbreak

\brak{b} $\frac{p^2+q^2cos\theta}{pcos\theta+qsin\theta}$

\brak{d} $\frac{(p^2+q^2)sin\theta}{(pcos\theta+qsin\theta)^2}$
    
\end{multicols}

\end{enumerate}

\end{document}
