\let\negmedspace\undefined
\let\negthickspace\undefined
\documentclass[journal]{IEEEtran}
\usepackage[a5paper, margin=10mm, onecolumn]{geometry}
%\usepackage{lmodern} % Ensure lmodern is loaded for pdflatex
\usepackage{tfrupee} % Include tfrupee package

\setlength{\headheight}{1cm} % Set the height of the header box
\setlength{\headsep}{0mm}     % Set the distance between the header box and the top of the text

\usepackage{gvv-book}
\usepackage{gvv}
\usepackage{cite}
\usepackage{amsmath,amssymb,amsfonts,amsthm}
\usepackage{algorithmic}
\usepackage{graphicx}
\usepackage{textcomp}
\usepackage{xcolor}
\usepackage{txfonts}
\usepackage{listings}
\usepackage{enumitem}
\usepackage{mathtools}
\usepackage{gensymb}
\usepackage{comment}
\usepackage[breaklinks=true]{hyperref}
\usepackage{tkz-euclide} 
\usepackage{listings}
% \usepackage{gvv}                                        
\def\inputGnumericTable{}                                 
\usepackage[latin1]{inputenc}                                
\usepackage{color}                                            
\usepackage{array}                                            
\usepackage{longtable}                                       
\usepackage{calc}                                             
\usepackage{multirow}                                         
\usepackage{hhline}                                           
\usepackage{ifthen}                                           
\usepackage{lscape}
\begin{document}
\bibliographystyle{IEEEtran}
\title{9.2.42}
\author{EE24BTECH11007 - Arnav Makarand Yadnopavit}
{\let\newpage\relax\maketitle}
\renewcommand{\thefigure}{\theenumi}
\renewcommand{\thetable}{\theenumi}
\setlength{\intextsep}{10pt} % Space between text and floats
\numberwithin{equation}{enumi}
\numberwithin{figure}{enumi}
\renewcommand{\thetable}{\theenumi}
\parindent 0px
Question:\\
The area of the region bounded by parabola $y^2 = x$ and the straight line $2y = x$ is\\
\solution
\begin{table}[h]
    \centering
    \begin{tabular}{|c|p{1cm}|p{7cm}|}
    \hline
	Variable  &  Value & Description\\
    \hline
        $V$ &  $\myvec{0&0\\0&1}$ & $x^\top Vx + 2u^\top x + f = 0$\\
    \hline
        $u$ & $\myvec{\frac{-1}{2}\\0}$ & $x^\top Vx + 2u^\top x + f = 0$\\
    \hline
	$f$ &   $0$ & $f$ $x^\top Vx + 2u^\top x + f = 0$\\
    \hline 
	$h$ &  $\myvec{0\\0}$ & Point on the given line\\
    \hline
        $m$ &  $\myvec{2 \\ 1}$ & Direction vector of given line\\
    \hline	
\end{tabular}

    \caption{Values}
    \label{tab:1}
\end{table}\\
The point of intersection of the line with the parabola is $x_i=h+k_i m$,\\
where,$k_i$ is a constant and is calculated as follows:-
\begin{align}
    k_i=\frac{1}{m^\top Vm}\brak{-m^\top \brak{Vh+u}\pm \sqrt{\sbrak{m^\top \brak{Vh+u}}^2-g\brak{h}\brak{m^\top Vm}}}
\end{align}
Substituting the input parameters in $k_i$,\\
We get,\\
\begin{align}
    k_i= 0, 2
\end{align}
Substituting $k_i$ in $x_i=h+k_i m$  we get,\\
\begin{align}
     x_1&=\myvec{0\\0}+\brak{0}\myvec{2\\1}\\
    \implies x_1 &=\myvec{0\\0} + \myvec{0\\0}\\
    \implies x_1 &=\myvec{0\\0}\\
    x_2 &=\myvec{0\\0}+\brak{2}\myvec{2\\1}\\
    \implies x_2&=\myvec{0\\0}+\myvec{4\\2}\\
    \implies x_2&=\myvec{4\\2}
\end{align}
The area bounded by the curve $y^2 = x$ and line $2y = x$ is given by
\begin{align}
    \int_{0}^{2} 2y-y^2 dy &= \brak{y^2-\frac{y^3}{3}}_{0}^{2} \\
    &= \brak{4-\frac{8}{3}}-0\\
    &= \frac{4}{3} 
\end{align}
The area of region bounded by the $y^2 = x$ and the straight line $2y = x$ is $\frac{4}{3}$
\begin{figure}[h]
    \centering
    \includegraphics[width=\columnwidth]{figs/fig.png}
    \caption{Plot of Parabola $y^2 = x$ along with the line $2y = x$}
 \end{figure}
\end{document}
