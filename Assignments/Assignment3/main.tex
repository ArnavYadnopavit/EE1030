\let\negmedspace\undefined
\let\negthickspace\undefined
\documentclass[journal]{IEEEtran}
\usepackage[a5paper, margin=10mm, onecolumn]{geometry}
%\usepackage{lmodern} % Ensure lmodern is loaded for pdflatex
\usepackage{tfrupee} % Include tfrupee package

\setlength{\headheight}{1cm} % Set the height of the header box
\setlength{\headsep}{0mm}     % Set the distance between the header box and the top of the text

\usepackage{gvv-book}
\usepackage{gvv}
\usepackage{cite}
\usepackage{amsmath,amssymb,amsfonts,amsthm}
\usepackage{algorithmic}
\usepackage{graphicx}
\usepackage{textcomp}
\usepackage{xcolor}
\usepackage{txfonts}
\usepackage{listings}
\usepackage{enumitem}
\usepackage{mathtools}
\usepackage{gensymb}
\usepackage{comment}
\usepackage[breaklinks=true]{hyperref}
\usepackage{tkz-euclide} 
\usepackage{listings}
% \usepackage{gvv}                                        
\def\inputGnumericTable{}                                 
\usepackage[latin1]{inputenc}                                
\usepackage{color}                                            
\usepackage{array}                                            
\usepackage{longtable}                                       
\usepackage{calc}                                             
\usepackage{multirow}                                         
\usepackage{hhline}                                           
\usepackage{ifthen}                                           
\usepackage{lscape}
\begin{document}
\bibliographystyle{IEEEtran}
\title{1.6.28}
\author{EE24BTECH11007 - Arnav Makarand Yadnopavit}
{\let\newpage\relax\maketitle}
\renewcommand{\thefigure}{\theenumi}
\renewcommand{\thetable}{\theenumi}
\setlength{\intextsep}{10pt} % Space between text and floats
\numberwithin{equation}{enumi}
\numberwithin{figure}{enumi}
\renewcommand{\thetable}{\theenumi}
\textbf{Question}:\\
Show that the points $\vec{A}$ \brak{-2\hat{i}+3\hat{j}+5\hat{k}}, $\vec{B}$ \brak{\hat{i}+2\hat{j}+3\hat{k}} and $\vec{C}$ \brak{7\hat{i}-\hat{k}} are collinear\\
\textbf{Solution:}
\begin{table}[h]
    \centering
    \begin{tabular}{|c|p{1cm}|p{7cm}|}
    \hline
	Variable  &  Value & Description\\
    \hline
        $V$ &  $\myvec{0&0\\0&1}$ & $x^\top Vx + 2u^\top x + f = 0$\\
    \hline
        $u$ & $\myvec{\frac{-1}{2}\\0}$ & $x^\top Vx + 2u^\top x + f = 0$\\
    \hline
	$f$ &   $0$ & $f$ $x^\top Vx + 2u^\top x + f = 0$\\
    \hline 
	$h$ &  $\myvec{0\\0}$ & Point on the given line\\
    \hline
        $m$ &  $\myvec{2 \\ 1}$ & Direction vector of given line\\
    \hline	
\end{tabular}

    \caption{Given Values}
    \label{tab:1}
\end{table}
From \tabref{tab:1}\\
The matrix
\begin{align}
\myvec{\vec{B}-\vec{A} & \vec{C}-\vec{A}}^\top=\myvec{3 & -1 & -2 \\ 9 & -3 & -6}\\
\xleftrightarrow[]{R_2=R_2-3R_1} 
\myvec{3 & -1 & -2 \\ 0 & 0 & 0}
\end{align}
which has rank 1.
\begin{figure}[h]
    \centering
    \includegraphics[width=\columnwidth]{figs/fig.png}
    \caption{Plot of $\vec{A}$,$\vec{B}$,$\vec{C}$}
 \end{figure}
\end{document}
